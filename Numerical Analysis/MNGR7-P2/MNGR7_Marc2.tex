\documentclass[a4paper,10.4pt]{article}

\usepackage[utf8]{inputenc}
\usepackage[catalan]{babel}
\usepackage[T1]{fontenc}
\usepackage{colortbl}
\usepackage{color}
\usepackage{amsfonts,amssymb,amsthm,amsmath}
\usepackage[margin=10pt,font=small,labelfont={sf,bf}]{caption}
\usepackage{appendix}
\usepackage{titling, titlesec}
\usepackage{graphicx}
\usepackage{multicol}
\usepackage{subcaption}
\usepackage{slantsc}
\usepackage{hyperref}
\usepackage{epstopdf}
\usepackage{fancyhdr}
\usepackage{units}
\usepackage[top=1.25in, bottom=1.25in, left=1.3in, right=1.3in]{geometry}


\setcounter{secnumdepth}{4}
\setlength{\parskip}{8pt}
\pagestyle{fancy}
\fancyhf{}
\fancyhead[R]{\small\sffamily Pràctica 1: Tema errors}
\fancyhead[L]{\sffamily Seminari Mètodes Numèrics}
\fancyfoot[C]{\thepage}
\titleformat{\section}{\Large\bfseries\sffamily}{\thesection}{1em}{}
\titleformat{\subsection}{\large\bfseries\sffamily}{\thesubsection}{1em}{}
\titleformat{\subsubsection}{\normalsize\bfseries\sffamily}{\thesubsubsection}{1em}{}

\newcommand{\N}{\mathbb{N}}
\newcommand{\Z}{\mathbb{Z}}
 
\begin{document}

 
\title{\Large\bfseries\sffamily Mètodes Numèrics\\Pràctica 1: \Large\sffamily Tema d'errors}
\author{\sffamily Marc Seguí Coll: Conclusions individuals\\\normalsize\emph{Universitat Autònoma de Barcelona}}
\date{\normalsize Març \emph{$2019$}}


\maketitle
%\epigraph{¿Qué es la vida? Una ilusión,
%una sombra, una ficción.}{\textit{Calderón de la Barca}}
 
\section{Importància en el condicionament de fórmules}

Moltes vegades es passa per alt que l'ordinador treballa amb la representació de nombres en punt flotant. Aquest fet no es pot ignorar ja que els errors comesos poden arribar a ser prou significatius sobretot en problemes mal condicionats.

Els errors moltes vegades vénen pel truncament/arrodoniment de les dades produït per la precisió finita limitada per la representació en punt flotant. Això no és problemàtic sempre i quan l'error relatiu associat a la fórmula emprada no es propagui fins a ordres de magnitud superiors. És en aquest punt on l'anàlisi numèric entra en joc i desenvolupa el seu paper més clau, buscar i definir nous mètodes per realitzar càlculs numèrics que redueixin l'error comès.

S'ha pogut observar a l'informe de la pràctica que una fórmula mal condicionada pot provocar que el resultat obtingut difereixi en diversos ordres de magnitud respecte el valor real. Com a cas extrem, al problema de la variança, que en ser el quadrat d'un nombre és positiva sempre, les operacions aritmètiques per calcular-la retornaven en un dels casos un nombre negatiu, degudes les cancel·lacions sofertes per arrodoniment de l'ordinador.

Per aquests motius és important conèixer bé les dades d'entrada a un programa/fórmula que vagi a realitzar la màquina i saber si el resultats obtinguts es veuran afectats per contaminació dels errors per culpa d'un mal condicionament del problema. En cas que no es coneguin les dades o l'error propagat sigui prou significatiu, convé controlar l'error comès i inclús ser capaç de trobar alternatives que permetin càlculs més segurs.

\section{Cancel·lacions}

Sens dubte alguns dels errors més freqüents que s'han trobat a la pràctica han sigut els de cancel·lació. Aquests succeeixen o bé per arrodoniment de les xifres (arrodonir en molts casos suposa la pèrdua de xifres significatives) o bé per operar nombres amb ordres de magnitud diferents. L'ordinador pot guardar el nombre com una mantissa amb quantitat limitada de xifres i la base en la qual està expressada el nombre elevada a un cert exponent, també limitat. Per això mateix en realitzar operacions amb nombres d'ordres de magnitud diferents, pot ser que es perdin xifres significatives en guardar el nombre resultant.

Els errors de cancel·lació no són greus si l'ordre de magnitud no canvia gaire entre el valor real i el resultat obtingut. El problema de les fórmules mal condicionades és que l'error comès per aquesta causa és normalment catastròfic perquè l'ordre de magnitud canvia significativament.   
\end{document}